% SPDX-License-Identifier: CC-BY-NC-SA-4.0
%
% submissions.tex -- submission guidelines
% Copyright (C) 2024  Jacob Koziej <jacobkoziej@gmail.com>

\documentclass{article}

% SPDX-License-Identifier: CC-BY-NC-SA-4.0
%
% preamble.tex -- document configuration
% Copyright (C) 2024  Jacob Koziej <jacobkoziej@gmail.com>

\usepackage{geometry}
\geometry{
	marginpar=1.8cm,
	paper=b5paper,
}

\usepackage{bookmark}
\usepackage{fancyhdr}
\usepackage{fontawesome}
\usepackage{marginnote}
\usepackage{titling}

\usepackage{mathtools}
\usepackage{unicode-math}
\unimathsetup{
	math-style=ISO,
	warnings-off={
		mathtools-colon,
		mathtools-overbracket,
	},
}

\usepackage{minted}
\setminted{
	breaklines,
	linenos,
	obeytabs,
}

\input{git-hash}

\author{Jacob~Koziej}

\pagestyle{fancy}
\setcounter{secnumdepth}{0}

\fancyhead{}
\fancyhead[L]{%
	\small\slshape%
	ECE 210: MATLAB Seminar -- Signals \& Systems%
}
\fancyhead[R]{%
	\scriptsize\ttfamily%
	\faCodeFork\,\gitHash\gitDirty%
}
\fancyfoot{}
\fancyfoot[L]{%
	\scriptsize\slshape%
	Copyright \copyright\ 2024 \theauthor%
	~---~%
	\ttfamily%
	\href{https://creativecommons.org/licenses/by-nc-sa/4.0/}%
	{CC BY-NC-SA 4.0}%
}
\fancyfoot[R]{%
	\scriptsize\ttfamily%
	\href{https://github.com/jacobkoziej/jk-ece210}%
	{github.com/jacobkoziej/jk-ece210}%
}
\RenewDocumentCommand{\footrule}{}{\rule{\headwidth}{\headrulewidth}}
\RenewDocumentCommand{\footruleskip}{}{1pt}

\NewDocumentCommand{\aside}{m}{%
	\marginnote{%
		\footnotesize%
		\raggedright%
		\textsl{\textbf{Aside---}}%
		\par\noindent%
		#1%
	}
}

\NewDocumentCommand{\footurl}{m}{\footnote{\url{#1}}}

\NewDocumentCommand{\mCommand}{om}{%
	\IfValueTF{#1} {%
		\href{#1}{\mintinline{matlab}{#2}}%
	} {%
		\mintinline{matlab}{#2}%
	}%
}

\NewDocumentCommand{\matrixField}{mmm}{%
	\symrm{M}_{#1 \times #2} (\symbb{#3})%
}

\NewDocumentCommand{\renderTitle}{}{%
	{\noindent\LARGE\scshape\thetitle\vspace{1ex}}%
	\pdfbookmark[1]{\thetitle}{title}%
}

\NewDocumentCommand{\vocab}{m}{\textsl{\textbf{#1}}}

\title{Submissions}
% SPDX-License-Identifier: CC-BY-NC-SA-4.0
%
% postamble.tex -- document configuration, continued...
% Copyright (C) 2024  Jacob Koziej <jacobkoziej@gmail.com>

\usepackage{hyperref}
\hypersetup{
	hidelinks,
	pdfinfo = {
		Title    = \thetitle,
		Author   = \theauthor,
		Subject  = MATLAB,
		Keywords = {MATLAB, programming},
	},
}


\begin{document}
\renderTitle

To streamline grading, I ask you submit assignments as described below.
Please follow the submission guidelines, as I \underline{will not grade}
work that does not meet these expectations!

\section{File Header}

Since one day you might be a professional engineer, it's important to
give people who read your code some basic information about your work. A
file header can aid in this:

\begin{minted}{matlab}
% SPDX-License-Identifier: GPL-3.0-or-later
%
% fib.m -- calculate the Nth Fibonacci number
% Copyright (C) 2024  Jacob Koziej <jacobkoziej@gmail.com>
\end{minted}

Notice the file name and short justification for the file's existence.
Also note the line with copyright information, but more importantly, a
contact email for desperate questions from junior engineers in a few
years.  Additionally, you can add licensing information if you plan to
share your code online.  If you're unsure about which license to choose,
I recommend checking out \url{https://choosealicense.com/} and then
finding the associated \href{https://spdx.org/licenses/}{\texttt{SPDX%
-License-Identifier}}.

\section{Code Normalization}

Normalize your code using \href{https://florianschanda.github.io/%
miss_hit/style_checker.html}{\texttt{mh\_style}}.  Doing so will ensure
a consistent style for code, making it easier to focus on the code
content over style.  There are instances where a code formatter's
decisions can make your code more difficult to read than a
non-standardized formatting approach.  In these rare cases, feel free to
add a MISS\_HIT pragma to ignore the line(s) in question.

\section{Printing Code}

Strangely enough, I would like paper submissions of your MATLAB code.
To aid in creating printable PDFs, I've written \href{https://github.%
com/jacobkoziej/jk-ece210/blob/master/bin/code2pdf}{\texttt{code2pdf}}.
The first argument should be the path to the input file.  The second
optional argument is the output file path.  If you don't specify an
output path, the script will default to generating a PDF in the same
base directory as the input file with the same base name.  Be careful,
as this command can be destructive!

\section{Saving Command Output}

To save the output of your MATLAB code, utilize the \mCommand[https://%
www.mathworks.com/help/matlab/ref/diary.html]{diary} command:

\begin{minted}{matlab}
diary output.txt
ScriptToRun
\end{minted}

MATLAB then saves any output from \mCommand{ScriptToRun} to
\mCommand{output.txt}.

\section{Saving Figures}

Part of the reason why people put up with MATLAB is because it can
generate good figures from performed calculations.  That said, they're
not of much use if condemned to the realm of MathWorks.  To create a PDF
with vectorized versions of all your figures, add the following snippet
to your code:

\begin{minted}{matlab}
OUTPUT = 'output.pdf';

r = groot;

delete(OUTPUT);
for i = numel(r.Children):-1:1
    exportgraphics(r.Children(i), OUTPUT, ...
                   'Append', true, 'ContentType', 'vector');
end
\end{minted}
\end{document}
