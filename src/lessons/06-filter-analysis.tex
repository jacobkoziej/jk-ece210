% SPDX-License-Identifier: CC-BY-NC-SA-4.0
%
% 06-filter-analysis.tex -- life in the frequency domain
% Copyright (C) 2024  Jacob Koziej <jacobkoziej@gmail.com>

\documentclass{article}

% SPDX-License-Identifier: CC-BY-NC-SA-4.0
%
% preamble.tex -- document configuration
% Copyright (C) 2024  Jacob Koziej <jacobkoziej@gmail.com>

\usepackage{geometry}
\geometry{
	marginpar=1.8cm,
	paper=b5paper,
}

\usepackage{bookmark}
\usepackage{fancyhdr}
\usepackage{fontawesome}
\usepackage{marginnote}
\usepackage{titling}

\usepackage{mathtools}
\usepackage{unicode-math}
\unimathsetup{
	math-style=ISO,
	warnings-off={
		mathtools-colon,
		mathtools-overbracket,
	},
}

\usepackage{minted}
\setminted{
	breaklines,
	linenos,
	obeytabs,
}

\input{git-hash}

\author{Jacob~Koziej}

\pagestyle{fancy}
\setcounter{secnumdepth}{0}

\fancyhead{}
\fancyhead[L]{%
	\small\slshape%
	ECE 210: MATLAB Seminar -- Signals \& Systems%
}
\fancyhead[R]{%
	\scriptsize\ttfamily%
	\faCodeFork\,\gitHash\gitDirty%
}
\fancyfoot{}
\fancyfoot[L]{%
	\scriptsize\slshape%
	Copyright \copyright\ 2024 \theauthor%
	~---~%
	\ttfamily%
	\href{https://creativecommons.org/licenses/by-nc-sa/4.0/}%
	{CC BY-NC-SA 4.0}%
}
\fancyfoot[R]{%
	\scriptsize\ttfamily%
	\href{https://github.com/jacobkoziej/jk-ece210}%
	{github.com/jacobkoziej/jk-ece210}%
}
\RenewDocumentCommand{\footrule}{}{\rule{\headwidth}{\headrulewidth}}
\RenewDocumentCommand{\footruleskip}{}{1pt}

\NewDocumentCommand{\aside}{m}{%
	\marginnote{%
		\footnotesize%
		\raggedright%
		\textsl{\textbf{Aside---}}%
		\par\noindent%
		#1%
	}
}

\NewDocumentCommand{\footurl}{m}{\footnote{\url{#1}}}

\NewDocumentCommand{\mCommand}{om}{%
	\IfValueTF{#1} {%
		\href{#1}{\mintinline{matlab}{#2}}%
	} {%
		\mintinline{matlab}{#2}%
	}%
}

\NewDocumentCommand{\matrixField}{mmm}{%
	\symrm{M}_{#1 \times #2} (\symbb{#3})%
}

\NewDocumentCommand{\renderTitle}{}{%
	{\noindent\LARGE\scshape\thetitle\vspace{1ex}}%
	\pdfbookmark[1]{\thetitle}{title}%
}

\NewDocumentCommand{\vocab}{m}{\textsl{\textbf{#1}}}

\title{Lesson 06: Filter Analysis}
% SPDX-License-Identifier: CC-BY-NC-SA-4.0
%
% postamble.tex -- document configuration, continued...
% Copyright (C) 2024  Jacob Koziej <jacobkoziej@gmail.com>

\usepackage{hyperref}
\hypersetup{
	hidelinks,
	pdfinfo = {
		Title    = \thetitle,
		Author   = \theauthor,
		Subject  = MATLAB,
		Keywords = {MATLAB, programming},
	},
}


\begin{document}
\renderTitle

Now that we're comfortable plotting, it's time we use our newfound skill
for something of high importance for Electrical Engineers: filter
analysis.  MATLAB offers a powerful Signal Processing Toolbox\footurl{%
https://www.mathworks.com/products/signal.html} that allows us to
perform this analysis with relative ease.

\section{The Fast Fourier Transform}

Since MATLAB exists in the \emph{real world}, we need to craft a Fourier
Transform that can exists in the confines of our measly computer.  We
define the multidimensional \vocab{Discrete Fourier Transform (DFT)}
with the following:

\begin{equation}
	X[\mat{k}]
	=
	\sum^{n - 1}_{\mat{m} = 0}
	x[\mat{m}]\,\e^{-\im 2\cpi \mat{k} \cdot \mat{m} / n}
\end{equation}

The equation above allows us to compute a \(n\)-point DFT of a
\(d\)-dimensional signal.  The value of \(n\) allows us to vary our
resolution in the frequency domain with so-called \vocab{bin
frequencies} with the following \vocab{bin spacing}:

\aside{Increasing \(n\) results in a higher resolution in the frequency
domain at the cost of computational time.}

\begin{equation}
	\Delta \omega = \frac{2\cpi}{n}
\end{equation}

The \vocab{Fast Fourier Transform (FFT)} is nothing more than an
\emph{optimization} of the DFT, going from \(O(n^2)\) to \(O(n\log n)\)
by removing redundant computations.\footnote{Veritasium made a great
video on the history behind the FFT that I highly recommend watching:
\url{https://www.youtube.com/watch?v=nmgFG7PUHfo}}  Today, almost
everyone implements the DFT as an FFT, and as such, you'll almost always
incorrectly hear the DFT referred to as the FFT!

In MATLAB, we can compute the FFT with the \mCommand[https://www.%
mathworks.com/help/matlab/ref/fft.html]{fft()} function:

\mCommandGraphic{06-filter-analysis.d/fft-shift}

Due to how the DFT is defined, the output of \mCommand[https://www.%
mathworks.com/help/matlab/ref/fft.html]{fft()} will be DC, followed by
positive and negative frequencies.  As such, it's typically desirable to
center our output around DC with \mCommand[https://www.mathworks.com/%
help/matlab/ref/fftshift.html]{fftshift()}.  Other things to note are
that the output may be complex-valued so we typically plot the magnitude
of the FFT and that the frequency range is from \(-f_s/2\) to \(f_s/2\).

\newpage

\section{The \(Z\)-Transform}

We define \vocab{the \(Z\)-Transform} as follows:

\begin{equation}
	X(z)
	\
	=
	\sum^{\infty}_{n = -\infty}
	x[n]z^{-n}
\end{equation}

The \(Z\)-Transform allows us to convert a discrete-time signal into a
complex-valued \(z\)-domain (or \(z\)-plane) representation.  This is
desirable as the transform domain gives us some nice properties:

\begin{itemize}
	\item
		\textbf{Linearity:} \(c_1 x_1  + c_2 x_2 \leftrightarrow
		c_1 X_1 + c_2 X_2\)

	\item
		\textbf{Delay:} \(x[n - 1] \leftrightarrow z^{-1}X(z)\)

	\item
		\textbf{Convolution:} \(h \ast x \leftrightarrow H \cdot
		X\)
\end{itemize}

The transfer domain also allows us to easily analyze the behavior of
systems in the form of polynomial transfer functions:

\begin{equation}
	H(z)
	=
	\frac{B(z)}{A(z)}
	=
	\frac
	{b_0 + b_1 z^{-1} + \dots + b_m z^{-m}}
	{a_0 + a_1 z^{-1} + \dots + a_n z^{-n}}
\end{equation}

\aside{This isn't purely a preference thing.  Analysis should \emph{%
never} be done in polynomial form until the last possible instance due
to the introduced floating-point quantization error!}

Although we can represent a polynomial transfer function in terms of its
coefficients, this is undesirable due to limited machine precision;
rather, we far prefer to keep transfer functions in pole-zero-gain form:

\begin{equation}
	H(z)
	=
	\frac{Z(z)}{P(z)}
	=
	K\frac
	{(z - z_1)^{n_1} \dots (z - z_N)^{n_N}}
	{(z - p_1)^{m_1} \dots (z - p_M)^{m_M}}
\end{equation}

where \(K\) is the gain of the system, the numerator contains the zeros
of the system, and the denominator contains the poles of the system.

\subsection{Conversion Between Polynomial and Pole-Zero-Gain Form}

In MATLAB, we store the polynomial coefficients as \textbf{row vectors}
and the poles, zeros, and gains of a system as \textbf{column vectors}.
It is important to remember this as the behavior of functions in the
Signals Processing Toolbox depend on this semantic difference!

Say we wanted to represent the following pole-zero-gain transfer
function:

\begin{equation}
	H(z)
	=
	\frac
	{(z + 3)(z - 1)^2}
	{z(z - 3)}
\end{equation}

We can represent it in MATLAB like so:

\begin{minted}{matlab}
z = [-3; 1; 1];
p = [0; 3];
k = 1;
\end{minted}

Then to convert this to a polynomial transfer function, we can call
\mCommand[https://www.mathworks.com/help/signal/ref/zp2tf.html]{%
zp2tf()}:

\mCommandOutput{06-filter-analysis.d/zpk2tf}

Conversely, we can represent the following polynomial transfer function:

\begin{equation}
	H(z)
	=
	\frac
	{1 + z^{-1} - 5z^{-2} + 3z^{-3}}
	{1 - 3z^{-1}}
\end{equation}

Like so in MATLAB:

\begin{minted}{matlab}
b = [1, 1, -5, 3];
a = [1, -3];
\end{minted}

\newpage

Then to convert this to a pole-zero-gain transfer function, we can call
\mCommand[https://www.mathworks.com/help/signal/ref/tf2zpk.html]{%
tf2zpk()}:

\aside{Be careful when using conversion functions, as some expect
positive power coefficients while others expect negative power
coefficients!}

\mCommandOutput{06-filter-analysis.d/ba2zpk}

\newpage

\subsection{Pole-Zero Plot}

To plot our transfer function, we can utilize \mCommand[https://www.%
mathworks.com/help/signal/ref/zplane.html]{zplane()}:

\mCommandGraphic{06-filter-analysis.d/zplane-plot}

Keep in mind that we can also call \mCommand[https://www.mathworks.com/%
help/signal/ref/zplane.html]{zplane()} with the \mCommand{b} and
\mCommand{a} vectors.

\subsection{Frequency Response}

To get the frequency response of a digital filter, we can call
\mCommand[https://www.mathworks.com/help/signal/ref/freqz.html]{%
freqz()}.  Although this function can plot the frequency response of a
digital filter, we can do better as we can tune the plots to our needs.

Note that when plotting the phase response of our filter, we use
\mCommand[https://www.mathworks.com/help/matlab/ref/unwrap.html]{%
unwrap()}, which will resolve jump discontinuities in our phase angle.
The following also covers the very basic functionality of \mCommand[%
https://www.mathworks.com/help/signal/ref/freqz.html]{freqz()}, as the
online documentation offers great examples for different use cases.

\mCommandGraphic{06-filter-analysis.d/freqz-plot}
\end{document}
