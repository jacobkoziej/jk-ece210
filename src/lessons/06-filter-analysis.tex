% SPDX-License-Identifier: CC-BY-NC-SA-4.0
%
% 06-filter-analysis.tex -- life in the frequency domain
% Copyright (C) 2024  Jacob Koziej <jacobkoziej@gmail.com>

\documentclass{article}

% SPDX-License-Identifier: CC-BY-NC-SA-4.0
%
% preamble.tex -- document configuration
% Copyright (C) 2024  Jacob Koziej <jacobkoziej@gmail.com>

\usepackage{geometry}
\geometry{
	marginpar=1.8cm,
	paper=b5paper,
}

\usepackage{bookmark}
\usepackage{fancyhdr}
\usepackage{fontawesome}
\usepackage{marginnote}
\usepackage{titling}

\usepackage{mathtools}
\usepackage{unicode-math}
\unimathsetup{
	math-style=ISO,
	warnings-off={
		mathtools-colon,
		mathtools-overbracket,
	},
}

\usepackage{minted}
\setminted{
	breaklines,
	linenos,
	obeytabs,
}

\input{git-hash}

\author{Jacob~Koziej}

\pagestyle{fancy}
\setcounter{secnumdepth}{0}

\fancyhead{}
\fancyhead[L]{%
	\small\slshape%
	ECE 210: MATLAB Seminar -- Signals \& Systems%
}
\fancyhead[R]{%
	\scriptsize\ttfamily%
	\faCodeFork\,\gitHash\gitDirty%
}
\fancyfoot{}
\fancyfoot[L]{%
	\scriptsize\slshape%
	Copyright \copyright\ 2024 \theauthor%
	~---~%
	\ttfamily%
	\href{https://creativecommons.org/licenses/by-nc-sa/4.0/}%
	{CC BY-NC-SA 4.0}%
}
\fancyfoot[R]{%
	\scriptsize\ttfamily%
	\href{https://github.com/jacobkoziej/jk-ece210}%
	{github.com/jacobkoziej/jk-ece210}%
}
\RenewDocumentCommand{\footrule}{}{\rule{\headwidth}{\headrulewidth}}
\RenewDocumentCommand{\footruleskip}{}{1pt}

\NewDocumentCommand{\aside}{m}{%
	\marginnote{%
		\footnotesize%
		\raggedright%
		\textsl{\textbf{Aside---}}%
		\par\noindent%
		#1%
	}
}

\NewDocumentCommand{\footurl}{m}{\footnote{\url{#1}}}

\NewDocumentCommand{\mCommand}{om}{%
	\IfValueTF{#1} {%
		\href{#1}{\mintinline{matlab}{#2}}%
	} {%
		\mintinline{matlab}{#2}%
	}%
}

\NewDocumentCommand{\matrixField}{mmm}{%
	\symrm{M}_{#1 \times #2} (\symbb{#3})%
}

\NewDocumentCommand{\renderTitle}{}{%
	{\noindent\LARGE\scshape\thetitle\vspace{1ex}}%
	\pdfbookmark[1]{\thetitle}{title}%
}

\NewDocumentCommand{\vocab}{m}{\textsl{\textbf{#1}}}

\title{Lesson 06: Filter Analysis}
% SPDX-License-Identifier: CC-BY-NC-SA-4.0
%
% postamble.tex -- document configuration, continued...
% Copyright (C) 2024  Jacob Koziej <jacobkoziej@gmail.com>

\usepackage{hyperref}
\hypersetup{
	hidelinks,
	pdfinfo = {
		Title    = \thetitle,
		Author   = \theauthor,
		Subject  = MATLAB,
		Keywords = {MATLAB, programming},
	},
}


\begin{document}
\renderTitle

Now that we're comfortable plotting, it's time we use our newfound skill
for something of high importance for Electrical Engineers: filter
analysis.  MATLAB offers a powerful Signal Processing Toolbox\footurl{%
https://www.mathworks.com/products/signal.html} that allows us to
perform this analysis with relative ease.

\section{The Fast Fourier Transform}

Since MATLAB exists in the \emph{real world}, we need to craft a Fourier
Transform that can exists in the confines of our measly computer.  We
define the multidimensional \vocab{Discrete Fourier Transform (DFT)}
with the following:

\begin{equation}
	X[\mat{k}]
	=
	\sum^{n - 1}_{\mat{m} = 0}
	x[\mat{m}]\,\e^{-\im 2\cpi \mat{k} \cdot \mat{m} / n}
\end{equation}

The equation above allows us to compute a \(n\)-point DFT of a
\(d\)-dimensional signal.  The value of \(n\) allows us to vary our
resolution in the frequency domain with so-called \vocab{bin
frequencies} with the following \vocab{bin spacing}:

\aside{Increasing \(n\) results in a higher resolution in the frequency
domain at the cost of computational time.}

\begin{equation}
	\Delta \omega = \frac{2\cpi}{n}
\end{equation}

The \vocab{Fast Fourier Transform (FFT)} is nothing more than an
\emph{optimization} of the DFT, going from \(O(n^2)\) to \(O(n\log n)\)
by removing redundant computations.\footnote{Veritasium made a great
video on the history behind the FFT that I highly recommend watching:
\url{https://www.youtube.com/watch?v=nmgFG7PUHfo}}  Today, almost
everyone implements the DFT as an FFT, and as such, you'll almost always
incorrectly hear the DFT referred to as the FFT!

In MATLAB, we can compute the FFT with the \mCommand[https://www.%
mathworks.com/help/matlab/ref/fft.html]{fft()} function:

\mCommandGraphic{06-filter-analysis.d/fft-shift}

Due to how the DFT is defined, the output of \mCommand[https://www.%
mathworks.com/help/matlab/ref/fft.html]{fft()} will be DC, followed by
positive and negative frequencies.  As such, it's typically desirable to
center our output around DC with \mCommand[https://www.mathworks.com/%
help/matlab/ref/fftshift.html]{fftshift()}.  Other things to note are
that the output may be complex-valued so we typically plot the magnitude
of the FFT and that the frequency range is from \(-f_s/2\) to \(f_s/2\).
\end{document}
