% SPDX-License-Identifier: CC-BY-NC-SA-4.0
%
% 10-image-compression.tex -- give me what's most important
% Copyright (C) 2024  Jacob Koziej <jacobkoziej@gmail.com>

\documentclass{article}

% SPDX-License-Identifier: CC-BY-NC-SA-4.0
%
% preamble.tex -- document configuration
% Copyright (C) 2024  Jacob Koziej <jacobkoziej@gmail.com>

\usepackage{geometry}
\geometry{
	marginpar=1.8cm,
	paper=b5paper,
}

\usepackage{bookmark}
\usepackage{fancyhdr}
\usepackage{fontawesome}
\usepackage{marginnote}
\usepackage{titling}

\usepackage{mathtools}
\usepackage{unicode-math}
\unimathsetup{
	math-style=ISO,
	warnings-off={
		mathtools-colon,
		mathtools-overbracket,
	},
}

\usepackage{minted}
\setminted{
	breaklines,
	linenos,
	obeytabs,
}

\input{git-hash}

\author{Jacob~Koziej}

\pagestyle{fancy}
\setcounter{secnumdepth}{0}

\fancyhead{}
\fancyhead[L]{%
	\small\slshape%
	ECE 210: MATLAB Seminar -- Signals \& Systems%
}
\fancyhead[R]{%
	\scriptsize\ttfamily%
	\faCodeFork\,\gitHash\gitDirty%
}
\fancyfoot{}
\fancyfoot[L]{%
	\scriptsize\slshape%
	Copyright \copyright\ 2024 \theauthor%
	~---~%
	\ttfamily%
	\href{https://creativecommons.org/licenses/by-nc-sa/4.0/}%
	{CC BY-NC-SA 4.0}%
}
\fancyfoot[R]{%
	\scriptsize\ttfamily%
	\href{https://github.com/jacobkoziej/jk-ece210}%
	{github.com/jacobkoziej/jk-ece210}%
}
\RenewDocumentCommand{\footrule}{}{\rule{\headwidth}{\headrulewidth}}
\RenewDocumentCommand{\footruleskip}{}{1pt}

\NewDocumentCommand{\aside}{m}{%
	\marginnote{%
		\footnotesize%
		\raggedright%
		\textsl{\textbf{Aside---}}%
		\par\noindent%
		#1%
	}
}

\NewDocumentCommand{\footurl}{m}{\footnote{\url{#1}}}

\NewDocumentCommand{\mCommand}{om}{%
	\IfValueTF{#1} {%
		\href{#1}{\mintinline{matlab}{#2}}%
	} {%
		\mintinline{matlab}{#2}%
	}%
}

\NewDocumentCommand{\matrixField}{mmm}{%
	\symrm{M}_{#1 \times #2} (\symbb{#3})%
}

\NewDocumentCommand{\renderTitle}{}{%
	{\noindent\LARGE\scshape\thetitle\vspace{1ex}}%
	\pdfbookmark[1]{\thetitle}{title}%
}

\NewDocumentCommand{\vocab}{m}{\textsl{\textbf{#1}}}

\title{Lesson 10: Image Compression}
% SPDX-License-Identifier: CC-BY-NC-SA-4.0
%
% postamble.tex -- document configuration, continued...
% Copyright (C) 2024  Jacob Koziej <jacobkoziej@gmail.com>

\usepackage{hyperref}
\hypersetup{
	hidelinks,
	pdfinfo = {
		Title    = \thetitle,
		Author   = \theauthor,
		Subject  = MATLAB,
		Keywords = {MATLAB, programming},
	},
}


\begin{document}
\renderTitle

\vocab{Compression} allows us to remove statistical redundancy from our
data to encode the same amount of information with fewer bits.
\vocab{Lossy compression} takes this one step further and drops
\enquote{less important} information to further reduce storage size.
We'll be exploring rudimentary lossy compression using the DFT.

\begin{figure}[ht!]
	\includegraphics[width=\textwidth]{10-image-compression.d/original.jpg}
	\caption{Jacob at the Summit of Half Dome}
\end{figure}

\section{Importing Images}

\mCommandCustom{10-image-compression.d/grayscale}{jpg}

Here we import the image into MATLAB with \mCommand[https://www.%
mathworks.com/help/matlab/ref/imread.html]{imread()}, convert it to
grayscale with \mCommand[https://www.mathworks.com/help/matlab/ref/%
rgb2gray.html]{rgb2gray()} to the simplify analysis, and write it to a
file with \mCommand[https://www.mathworks.com/help/matlab/ref/imwrite.%
html]{imwrite()}.

\section{2-D DFT}

\mCommandGraphic{10-image-compression.d/gray-fft2}

Here, we take the 2-D~DFT with \mCommand[https://www.mathworks.com/%
help/matlab/ref/fft2.html]{fft2()} and use \mCommand[https://www.%
mathworks.com/help/matlab/ref/imshow.html]{imshow()} to display its
output.
\end{document}
