% SPDX-License-Identifier: CC-BY-NC-SA-4.0
%
% 08-symbolic-math.tex -- your math teacher's worst nightmare
% Copyright (C) 2024  Jacob Koziej <jacobkoziej@gmail.com>

\documentclass{article}

% SPDX-License-Identifier: CC-BY-NC-SA-4.0
%
% preamble.tex -- document configuration
% Copyright (C) 2024  Jacob Koziej <jacobkoziej@gmail.com>

\usepackage{geometry}
\geometry{
	marginpar=1.8cm,
	paper=b5paper,
}

\usepackage{bookmark}
\usepackage{fancyhdr}
\usepackage{fontawesome}
\usepackage{marginnote}
\usepackage{titling}

\usepackage{mathtools}
\usepackage{unicode-math}
\unimathsetup{
	math-style=ISO,
	warnings-off={
		mathtools-colon,
		mathtools-overbracket,
	},
}

\usepackage{minted}
\setminted{
	breaklines,
	linenos,
	obeytabs,
}

\input{git-hash}

\author{Jacob~Koziej}

\pagestyle{fancy}
\setcounter{secnumdepth}{0}

\fancyhead{}
\fancyhead[L]{%
	\small\slshape%
	ECE 210: MATLAB Seminar -- Signals \& Systems%
}
\fancyhead[R]{%
	\scriptsize\ttfamily%
	\faCodeFork\,\gitHash\gitDirty%
}
\fancyfoot{}
\fancyfoot[L]{%
	\scriptsize\slshape%
	Copyright \copyright\ 2024 \theauthor%
	~---~%
	\ttfamily%
	\href{https://creativecommons.org/licenses/by-nc-sa/4.0/}%
	{CC BY-NC-SA 4.0}%
}
\fancyfoot[R]{%
	\scriptsize\ttfamily%
	\href{https://github.com/jacobkoziej/jk-ece210}%
	{github.com/jacobkoziej/jk-ece210}%
}
\RenewDocumentCommand{\footrule}{}{\rule{\headwidth}{\headrulewidth}}
\RenewDocumentCommand{\footruleskip}{}{1pt}

\NewDocumentCommand{\aside}{m}{%
	\marginnote{%
		\footnotesize%
		\raggedright%
		\textsl{\textbf{Aside---}}%
		\par\noindent%
		#1%
	}
}

\NewDocumentCommand{\footurl}{m}{\footnote{\url{#1}}}

\NewDocumentCommand{\mCommand}{om}{%
	\IfValueTF{#1} {%
		\href{#1}{\mintinline{matlab}{#2}}%
	} {%
		\mintinline{matlab}{#2}%
	}%
}

\NewDocumentCommand{\matrixField}{mmm}{%
	\symrm{M}_{#1 \times #2} (\symbb{#3})%
}

\NewDocumentCommand{\renderTitle}{}{%
	{\noindent\LARGE\scshape\thetitle\vspace{1ex}}%
	\pdfbookmark[1]{\thetitle}{title}%
}

\NewDocumentCommand{\vocab}{m}{\textsl{\textbf{#1}}}

\title{Lesson 08: Symbolic Math}
% SPDX-License-Identifier: CC-BY-NC-SA-4.0
%
% postamble.tex -- document configuration, continued...
% Copyright (C) 2024  Jacob Koziej <jacobkoziej@gmail.com>

\usepackage{hyperref}
\hypersetup{
	hidelinks,
	pdfinfo = {
		Title    = \thetitle,
		Author   = \theauthor,
		Subject  = MATLAB,
		Keywords = {MATLAB, programming},
	},
}


\begin{document}
\renderTitle

Most times, when we use MATLAB, we're interested in performing numerical
calculations, but what if we wanted to determine analytic solutions to
problems?  To do this, we can utilize the Symbolic Math Toolbox%
\footurl{https://www.mathworks.com/products/symbolic.html} and turn
tedious math homework into a thing of the past!\footnote{Disclaimer!
I am not condoning robbing yourself of valuable practice!}

\section{Numerical Stability}

We know from MA~111 that we can define the natural logarithm like so:

\begin{equation}
	\ln(x)
	=
	\int^x_1
	\frac{1}{t}
	dt
\end{equation}

However, \(1 / t\) can become problematic when dealing with large
floating point values.  Consider what happens when we try to compute
\(\ln\left(1 \times 10^{32}\right)\) using this definition numerically:

\mCommandOutput{08-symbolic-math.d/numerical}

What we've run into is an issue with the \vocab{numerical stability} of
floating point values.  You'll quickly learn that many approaches to
solving a problem may result in numerical instability.  There are
typically a few ways of solving this issue: we can use fixed point
arithmetic that gives us the precision we need, use a different
numerical approximation, or use an analytic solution and evaluate it to
the precision we desire.

\section{Symbolic Expressions}

We can create symbolic variables with
\mCommand[https://www.mathworks.com/help/symbolic/syms.html]{syms}:

\mCommandOutput{08-symbolic-math.d/simple}

We can then substitute values into the expression:

\mCommandOutput{08-symbolic-math.d/subst}

We can also create symbolic functions:

\mCommandOutput{08-symbolic-math.d/func}

And we can evaluate them:

\mCommandOutput{08-symbolic-math.d/func-eval}

\section{Solving and Manipulation}

Let's start by solving symbolic expression for a variable:

\mCommandOutput{08-symbolic-math.d/simple-solve}

We can also add assumptions about our variables for MATLAB to respect:

\mCommandOutput{08-symbolic-math.d/assume-solve}

We can even solve a system of equations:

\mCommandOutput{08-symbolic-math.d/sys-solve}

The \mCommand[https://www.mathworks.com/help/symbolic/sym.solve.html]{%
solve()} function is but one of many symbolic solvers available in
MATLAB.  I encourage you to explore other available solvers to pick the
right one for your application: \url{https://www.mathworks.com/help/%
symbolic/equation-solving.html}

We can also \mCommand[https://www.mathworks.com/help/symbolic/simplify.%
html]{simplify()} symbolic expressions:

\mCommandOutput{08-symbolic-math.d/simple-simplify}

\newpage

Note that \mCommand[https://www.mathworks.com/help/symbolic/simplify.%
html]{simplify()} will try its best to simplify an expression, but it
might not always do what you want.  Again, MATLAB offers a wide variety
of functions to manipulate symbolic expressions, and it is up to you to
explore them to pick the right one for your application: \url{https://%
www.mathworks.com/help/symbolic/formula-manipulation-and-%
simplification.html}

\section{Evaluation}

Let's go back to our original problem and evaluate it symbolically:

\mCommandOutput{08-symbolic-math.d/symbolic}

Notice how MATLAB correctly simplified this integral into an equivalent
symbolic expression!  And to get a floating point answer, we need to
convert it to a \mCommand[https://www.mathworks.com/help/matlab/ref/%
double.html]{double}:

\mCommandOutput{08-symbolic-math.d/double-ans}

But we can actually evaluate the expression to an arbitrary precision
with \mCommand[https://www.mathworks.com/help/symbolic/vpa.html]{vpa()}:

\mCommandOutput{08-symbolic-math.d/vpa-ans}
\end{document}
