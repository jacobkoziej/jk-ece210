% SPDX-License-Identifier: CC-BY-NC-SA-4.0
%
% 07-filter-design.tex -- living by the hippocratic oath
% Copyright (C) 2024  Jacob Koziej <jacobkoziej@gmail.com>

\documentclass{article}

% SPDX-License-Identifier: CC-BY-NC-SA-4.0
%
% preamble.tex -- document configuration
% Copyright (C) 2024  Jacob Koziej <jacobkoziej@gmail.com>

\usepackage{geometry}
\geometry{
	marginpar=1.8cm,
	paper=b5paper,
}

\usepackage{bookmark}
\usepackage{fancyhdr}
\usepackage{fontawesome}
\usepackage{marginnote}
\usepackage{titling}

\usepackage{mathtools}
\usepackage{unicode-math}
\unimathsetup{
	math-style=ISO,
	warnings-off={
		mathtools-colon,
		mathtools-overbracket,
	},
}

\usepackage{minted}
\setminted{
	breaklines,
	linenos,
	obeytabs,
}

\input{git-hash}

\author{Jacob~Koziej}

\pagestyle{fancy}
\setcounter{secnumdepth}{0}

\fancyhead{}
\fancyhead[L]{%
	\small\slshape%
	ECE 210: MATLAB Seminar -- Signals \& Systems%
}
\fancyhead[R]{%
	\scriptsize\ttfamily%
	\faCodeFork\,\gitHash\gitDirty%
}
\fancyfoot{}
\fancyfoot[L]{%
	\scriptsize\slshape%
	Copyright \copyright\ 2024 \theauthor%
	~---~%
	\ttfamily%
	\href{https://creativecommons.org/licenses/by-nc-sa/4.0/}%
	{CC BY-NC-SA 4.0}%
}
\fancyfoot[R]{%
	\scriptsize\ttfamily%
	\href{https://github.com/jacobkoziej/jk-ece210}%
	{github.com/jacobkoziej/jk-ece210}%
}
\RenewDocumentCommand{\footrule}{}{\rule{\headwidth}{\headrulewidth}}
\RenewDocumentCommand{\footruleskip}{}{1pt}

\NewDocumentCommand{\aside}{m}{%
	\marginnote{%
		\footnotesize%
		\raggedright%
		\textsl{\textbf{Aside---}}%
		\par\noindent%
		#1%
	}
}

\NewDocumentCommand{\footurl}{m}{\footnote{\url{#1}}}

\NewDocumentCommand{\mCommand}{om}{%
	\IfValueTF{#1} {%
		\href{#1}{\mintinline{matlab}{#2}}%
	} {%
		\mintinline{matlab}{#2}%
	}%
}

\NewDocumentCommand{\matrixField}{mmm}{%
	\symrm{M}_{#1 \times #2} (\symbb{#3})%
}

\NewDocumentCommand{\renderTitle}{}{%
	{\noindent\LARGE\scshape\thetitle\vspace{1ex}}%
	\pdfbookmark[1]{\thetitle}{title}%
}

\NewDocumentCommand{\vocab}{m}{\textsl{\textbf{#1}}}

\title{Lesson 07: Filter Design}
% SPDX-License-Identifier: CC-BY-NC-SA-4.0
%
% postamble.tex -- document configuration, continued...
% Copyright (C) 2024  Jacob Koziej <jacobkoziej@gmail.com>

\usepackage{hyperref}
\hypersetup{
	hidelinks,
	pdfinfo = {
		Title    = \thetitle,
		Author   = \theauthor,
		Subject  = MATLAB,
		Keywords = {MATLAB, programming},
	},
}


\begin{document}
\renderTitle

It's time we abide by the Hippocratic Oath of Signal Processing%
\footnote{Do no harm to the signal.  Anything you do will harm the
signal.} and start to design filters.  Designing a filter may seem
daunting, but luckily for us, MATLAB provides us with easy-to-use
digital filter design tools,\footurl{https://www.mathworks.com/help/%
signal/filter-design.html} simplifying the process to nothing more than
calling functions with the appropriate arguments.

\section{Design Considerations}

As an engineer, it is your job to determine the appropriate trade-offs
when designing a filter for an application, as it really does come down
to using the \enquote{right} tool for the job.  For now, let's consider
only the following types of IIR filters:

\begin{longtable}{lll}
\toprule
Name & Passband & Stopband \\
\midrule
Butterworth  & Monotonic  & Monotonic \\
Chebyshev I  & Equiripple & Monotonic \\
Chebyshev II & Monotonic  & Equiripple \\
Elliptic     & Equiripple & Equiripple \\
\bottomrule
\\
\caption{IIR Filter Types}
\end{longtable}

Given the table above, we can select a filter type that will suit our
needs.  For example, we'd use a Butterworth filter in an audio
application as we often desire monotonic behavior.  After all, we
wouldn't want our filter to vary the attenuation of frequencies in our
passband and stopband, as that would sound awful.

Once we've decided on a filter type, we need to specify a few
properties:

\begin{itemize}
	\item
		Filter order

	\item
		Peak-to-peak passband ripple

	\item
		Stopband attenuation

	\item
		Passband/Stopband edge frequencies

	\item
		Size of the transition band
\end{itemize}

That said, for the rest of the lesson we'll be designing a Chebyshev II
bandstop filter as the process is largely the same for the other filter
types.

\section{Determining Filter Order}

When designing a filter, we often specify all the properties but the
order and leave that as a final step for MATLAB.

\mCommandOutput{07-filter-design.d/ord}

Here, we're designing a filter to operate at 44.1~kHz, with passband
corner frequencies at 1.0 and 19.0~kHz, along with stopband corner
frequencies at 2.2 and 16.8~kHz normalized to the Nyquist frequency.
We're also allowing for 5~dB of peak-to-peak ripple in the passband and
we're mandating the stopband be attenuated by 40~dB.

An important thing to note about MATLAB order functions for
bandpass/bandstop filters is that the returned order of the prototype
filter is \textbf{half} of the actual order.  The reason for this is
quite simple: a bandpass/bandstop filter is composed of a low-pass and
high-pass filter!

\newpage

\section{Filter Generation}

Once we have the order, we are ready to generate the filter:

\mCommandOutput{07-filter-design.d/gen}

Note that this step largely depends on the filter you're trying to
generate; hence, you must reference the corresponding MATLAB
documentation.  Also, note that we're also using the zero-pole-gain
variant of the output to avoid numerical instability that comes with a
polynomial transfer function.

\begin{minipage}{\textwidth}
	We can quickly confirm that we got our desired filter with a
	quick call to \mCommand[https://www.mathworks.com/help/signal/%
	ref/freqz.html]{freqz()}:

	\vspace{1em}

	\mCommandGraphic{07-filter-design.d/response}
\end{minipage}
\end{document}
