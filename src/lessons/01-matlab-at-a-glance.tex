% SPDX-License-Identifier: CC-BY-NC-SA-4.0
%
% 01-matlab-at-a-glance.tex -- getting on your feet in MATLAB
% Copyright (C) 2024  Jacob Koziej <jacobkoziej@gmail.com>

\documentclass{article}

% SPDX-License-Identifier: CC-BY-NC-SA-4.0
%
% preamble.tex -- document configuration
% Copyright (C) 2024  Jacob Koziej <jacobkoziej@gmail.com>

\usepackage{geometry}
\geometry{
	paper=b5paper,
}

\usepackage{bookmark}
\usepackage{fancyhdr}
\usepackage{fontawesome}
\usepackage{titling}

\usepackage{mathtools}
\usepackage{unicode-math}
\unimathsetup{
	math-style=ISO,
	warnings-off={
		mathtools-colon,
		mathtools-overbracket,
	},
}

\usepackage{minted}
\setminted{
	breaklines,
	linenos,
	obeytabs,
}

\input{git-hash}

\author{Jacob~Koziej}

\pagestyle{fancy}
\setcounter{secnumdepth}{0}

\fancyhead{}
\fancyhead[L]{%
	\small\slshape%
	ECE 210: MATLAB Seminar -- Signals \& Systems%
}
\fancyhead[R]{%
	\scriptsize\ttfamily%
	\faCodeFork\,\gitHash\gitDirty%
}
\fancyfoot{}
\fancyfoot[L]{%
	\scriptsize\slshape%
	Copyright \copyright\ 2024 \theauthor%
	~---~%
	\ttfamily%
	\href{https://creativecommons.org/licenses/by-nc-sa/4.0/}%
	{CC BY-NC-SA 4.0}%
}
\fancyfoot[R]{%
	\scriptsize\ttfamily%
	\href{https://github.com/jacobkoziej/jk-ece210}%
	{github.com/jacobkoziej/jk-ece210}%
}
\RenewDocumentCommand{\footrule}{}{\rule{\headwidth}{\headrulewidth}}
\RenewDocumentCommand{\footruleskip}{}{1pt}

\NewDocumentCommand{\footurl}{m}{\footnote{\url{#1}}}

\NewDocumentCommand{\mCommand}{om}{%
	\IfValueTF{#1} {%
		\href{#1}{\mintinline{matlab}{#2}}%
	} {%
		\mintinline{matlab}{#2}%
	}%
}

\NewDocumentCommand{\renderTitle}{}{%
	{\noindent\LARGE\scshape\thetitle\vspace{1ex}}%
	\pdfbookmark[1]{\thetitle}{title}%
}

\NewDocumentCommand{\vocab}{m}{\textsl{\textbf{#1}}}

\title{Lesson 01: MATLAB at a Glance}
% SPDX-License-Identifier: CC-BY-NC-SA-4.0
%
% hyperref.tex -- hyperref setup
% Copyright (C) 2024  Jacob Koziej <jacobkoziej@gmail.com>

\usepackage{hyperref}
\hypersetup{
	colorlinks,
	linkcolor=black,
	pdfinfo = {
		Title    = \thetitle,
		Author   = \theauthor,
		Subject  = MATLAB,
		Keywords = {MATLAB, programming},
	},
}


\begin{document}
\renderTitle

\noindent
MATLAB can be unbelievably \emph{frustrating...}

\

\noindent
I have lost count of the number of times this program has caused me
immeasurable amounts of pain through no fault of my own. With poor
language design, undocumented quirks, and, at times, seemingly
non-deterministic behavior, MATLAB is a beast to tame.  It will
shamelessly stab you when you least expect it and show you absolutely
zero remorse!

\

\noindent
That said, it can be incredibly useful when the stars align, so here we
are...

\section{The MATLAB Environment}

Before we jump head first into the fundamentals of MATLAB, it helps if
we take a step back and go over how to interact with its environment.

\subsection{So What’s an M-File?}

\vocab{M-files}, denoted with a \texttt{.m} extension, are the primary
method for scripting MATLAB operations.  MATLAB executes the contents of
these files as if you had typed them explicitly into the command prompt.
Confusingly, M-files can be both executable user scripts or define
functions, which we can then reference in other M-files.  To execute the
contents of an M-file, type its name into the command prompt without the
\texttt{.m} extension.
\end{document}
