% SPDX-License-Identifier: CC-BY-NC-SA-4.0
%
% 01-matlab-at-a-glance.tex -- getting on your feet in MATLAB
% Copyright (C) 2024  Jacob Koziej <jacobkoziej@gmail.com>

\documentclass{article}

% SPDX-License-Identifier: CC-BY-NC-SA-4.0
%
% preamble.tex -- document configuration
% Copyright (C) 2024  Jacob Koziej <jacobkoziej@gmail.com>

\usepackage{geometry}
\geometry{
	marginpar=1.8cm,
	paper=b5paper,
}

\usepackage{bookmark}
\usepackage{fancyhdr}
\usepackage{fontawesome}
\usepackage{marginnote}
\usepackage{titling}

\usepackage{mathtools}
\usepackage{unicode-math}
\unimathsetup{
	math-style=ISO,
	warnings-off={
		mathtools-colon,
		mathtools-overbracket,
	},
}

\usepackage{minted}
\setminted{
	breaklines,
	linenos,
	obeytabs,
}

\input{git-hash}

\author{Jacob~Koziej}

\pagestyle{fancy}
\setcounter{secnumdepth}{0}

\fancyhead{}
\fancyhead[L]{%
	\small\slshape%
	ECE 210: MATLAB Seminar -- Signals \& Systems%
}
\fancyhead[R]{%
	\scriptsize\ttfamily%
	\faCodeFork\,\gitHash\gitDirty%
}
\fancyfoot{}
\fancyfoot[L]{%
	\scriptsize\slshape%
	Copyright \copyright\ 2024 \theauthor%
	~---~%
	\ttfamily%
	\href{https://creativecommons.org/licenses/by-nc-sa/4.0/}%
	{CC BY-NC-SA 4.0}%
}
\fancyfoot[R]{%
	\scriptsize\ttfamily%
	\href{https://github.com/jacobkoziej/jk-ece210}%
	{github.com/jacobkoziej/jk-ece210}%
}
\RenewDocumentCommand{\footrule}{}{\rule{\headwidth}{\headrulewidth}}
\RenewDocumentCommand{\footruleskip}{}{1pt}

\NewDocumentCommand{\aside}{m}{%
	\marginnote{%
		\footnotesize%
		\raggedright%
		\textsl{\textbf{Aside---}}%
		\par\noindent%
		#1%
	}
}

\NewDocumentCommand{\footurl}{m}{\footnote{\url{#1}}}

\NewDocumentCommand{\mCommand}{om}{%
	\IfValueTF{#1} {%
		\href{#1}{\mintinline{matlab}{#2}}%
	} {%
		\mintinline{matlab}{#2}%
	}%
}

\NewDocumentCommand{\matrixField}{mmm}{%
	\symrm{M}_{#1 \times #2} (\symbb{#3})%
}

\NewDocumentCommand{\renderTitle}{}{%
	{\noindent\LARGE\scshape\thetitle\vspace{1ex}}%
	\pdfbookmark[1]{\thetitle}{title}%
}

\NewDocumentCommand{\vocab}{m}{\textsl{\textbf{#1}}}

\title{Lesson 01: MATLAB at a Glance}
% SPDX-License-Identifier: CC-BY-NC-SA-4.0
%
% hyperref.tex -- hyperref setup
% Copyright (C) 2024  Jacob Koziej <jacobkoziej@gmail.com>

\usepackage{hyperref}
\hypersetup{
	colorlinks,
	linkcolor=black,
	pdfinfo = {
		Title    = \thetitle,
		Author   = \theauthor,
		Subject  = MATLAB,
		Keywords = {MATLAB, programming},
	},
}


\begin{document}
\renderTitle

\noindent
MATLAB can be unbelievably \emph{frustrating...}

\

\noindent
I have lost count of the number of times this program has caused me
immeasurable amounts of pain through no fault of my own. With poor
language design, undocumented quirks, and, at times, seemingly
non-deterministic behavior, MATLAB is a beast to tame.  It will
shamelessly stab you when you least expect it and show you absolutely
zero remorse!

\

\noindent
That said, it can be incredibly useful when the stars align, so here we
are...

\section{The MATLAB Environment}

Before we jump head first into the fundamentals of MATLAB, it helps if
we take a step back and go over how to interact with its environment.

\subsection{So What’s an M-File?}

\vocab{M-files}, denoted with a \texttt{.m} extension, are the primary
method for scripting MATLAB operations.  MATLAB executes the contents of
these files as if you had typed them explicitly into the command prompt.
Confusingly, M-files can be both executable user scripts or define
functions, which we can then reference in other M-files.  To execute the
contents of an M-file, type its name into the command prompt without the
\texttt{.m} extension.

\subsection{But Where to Find M-Files?}

\vocab{The MATLAB Search Path}\footurl{https://www.mathworks.com/help/%
matlab/matlab_env/what-is-the-matlab-search-path.html} specifies the
paths MATLAB will search for M-files.  It will first search the current
working directory (accessible with \mCommand[https://www.mathworks.com/%
help/matlab/ref/pwd.html]{pwd}), your \mCommand[https://www.mathworks%
.com/help/matlab/ref/userpath.html]{userpath}, paths specified in the
\texttt{MATLABPATH} environment variable, and finally, paths added by
installed packages.  To view the paths in the current search path, you
can use the \mCommand[https://www.mathworks.com/help/matlab/ref/path.%
html]{path} command.  To locate the path of an M-file, use the
\mCommand[https://www.mathworks.com/help/matlab/ref/which.html]{which}
command followed by the name of the M-file you're trying to find.

\subsection{User-Defined Startup Script}

If you'd like to execute some commands before MATLAB starts, you can add
a \mCommand[https://www.mathworks.com/help/matlab/ref/startup.html]%
{startup} M-file in the search path.  This file is handy if you wish to
modify the environment at runtime (i.e. set graphic options).  Feel free
to take a look at my \href{https://github.com/jacobkoziej/dotfiles/%
blob/master/.config/matlab/startup.m}{\texttt{startup.m}} if you'd like
to see an example.

\section{Fundamentals}

Now that we have the \emph{slightest} idea of what's going on, it's out
the frying pan and into the fire!

\subsection{Everything's a Matrix!}

Who would have thought that MATrix LABratory represents its data as
matrices?  Matrices generally have three forms:

\begin{itemize}
	\item
		Matrices of the form \(\matrixField{m}{n}{F}\)

	\item
		Vectors of the form \(\matrixField{m}{1}{F}\) or
		\(\matrixField{1}{n}{F}\)

	\item
		Scalars of the form \(\matrixField{1}{1}{F}\)
\end{itemize}

\noindent
Certain MATLAB commands may want a specific type of matrix (i.e. a row
vector) or may operate on different portions of a matrix (i.e. its
columns).  We will see examples of this later.  For now, let's go over
how you might instantiate different types of matrices:\aside{Well,
actually, these are all represented as N-dimensional arrays with an
infinite amount of singleton dimensions.\footnotemark}\footnotetext{%
\url{https://blogs.mathworks.com/loren/2006/08/09/essence-of-indexing/}}

\inputminted{matlab}{01-matlab-at-a-glance.d/instantiate-matrix.m}
\end{document}
