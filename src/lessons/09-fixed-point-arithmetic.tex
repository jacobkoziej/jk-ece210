% SPDX-License-Identifier: CC-BY-NC-SA-4.0
%
% 09-fixed-point-arithmetic.tex -- precision? never head of her
% Copyright (C) 2024  Jacob Koziej <jacobkoziej@gmail.com>

\documentclass{article}

% SPDX-License-Identifier: CC-BY-NC-SA-4.0
%
% preamble.tex -- document configuration
% Copyright (C) 2024  Jacob Koziej <jacobkoziej@gmail.com>

\usepackage{geometry}
\geometry{
	paper=b5paper,
}

\usepackage{bookmark}
\usepackage{fancyhdr}
\usepackage{fontawesome}
\usepackage{titling}

\usepackage{mathtools}
\usepackage{unicode-math}
\unimathsetup{
	math-style=ISO,
	warnings-off={
		mathtools-colon,
		mathtools-overbracket,
	},
}

\usepackage{minted}
\setminted{
	breaklines,
	linenos,
	obeytabs,
}

\input{git-hash}

\author{Jacob~Koziej}

\pagestyle{fancy}
\setcounter{secnumdepth}{0}

\fancyhead{}
\fancyhead[L]{%
	\small\slshape%
	ECE 210: MATLAB Seminar -- Signals \& Systems%
}
\fancyhead[R]{%
	\scriptsize\ttfamily%
	\faCodeFork\,\gitHash\gitDirty%
}
\fancyfoot{}
\fancyfoot[L]{%
	\scriptsize\slshape%
	Copyright \copyright\ 2024 \theauthor%
	~---~%
	\ttfamily%
	\href{https://creativecommons.org/licenses/by-nc-sa/4.0/}%
	{CC BY-NC-SA 4.0}%
}
\fancyfoot[R]{%
	\scriptsize\ttfamily%
	\href{https://github.com/jacobkoziej/jk-ece210}%
	{github.com/jacobkoziej/jk-ece210}%
}
\RenewDocumentCommand{\footrule}{}{\rule{\headwidth}{\headrulewidth}}
\RenewDocumentCommand{\footruleskip}{}{1pt}

\NewDocumentCommand{\footurl}{m}{\footnote{\url{#1}}}

\NewDocumentCommand{\mCommand}{om}{%
	\IfValueTF{#1} {%
		\href{#1}{\mintinline{matlab}{#2}}%
	} {%
		\mintinline{matlab}{#2}%
	}%
}

\NewDocumentCommand{\renderTitle}{}{%
	{\noindent\LARGE\scshape\thetitle\vspace{1ex}}%
	\pdfbookmark[1]{\thetitle}{title}%
}

\NewDocumentCommand{\vocab}{m}{\textsl{\textbf{#1}}}

\title{Lesson 09: Fixed-Point Arithmetic}
% SPDX-License-Identifier: CC-BY-NC-SA-4.0
%
% postamble.tex -- document configuration, continued...
% Copyright (C) 2024  Jacob Koziej <jacobkoziej@gmail.com>

\usepackage{hyperref}
\hypersetup{
	hidelinks,
	pdfinfo = {
		Title    = \thetitle,
		Author   = \theauthor,
		Subject  = MATLAB,
		Keywords = {MATLAB, programming},
	},
}


\begin{document}
\renderTitle

One day, you may have the great fortune of implementing a digital filter
on an embedded device where you only have a few microseconds to filter
before the next sample.  In such a scenario, performing floating-point
operations is impractical due to how much slower they are to \vocab{%
Multiply–Accumulate (MAC)} operations.  To benefit from MAC operations,
we turn to fixed-point arithmetic, but in the process, we open Pandora's
box and must account for the substantial quantization erorr we've
introduced.  Luckily for us, MATLAB offers us a Fixed-Point Designer%
\footurl{https://www.mathworks.com/help/fixedpoint/index.html} for
optimizing and implementing fixed-point algorithms.

\section{Constructing Fixed-Point Types}

\begin{figure}[!ht]
\Centering
\small
\begin{tabular}{c}
	\lstinputlisting[basicstyle=\ttfamily]%
	{09-fixed-point-arithmetic.d/fixed-point.txt}
\end{tabular}
\caption{Fixed-Point Value}
\end{figure}

First, we must tell MATLAB the type of fixed-point value we wish to
construct:

\mCommandOutput{09-fixed-point-arithmetic.d/fixed-point-type}

In our case, we've constructed a signed 32-bit fixed-point value with a
fraction length of 30 bits, giving us a range from -2 to \(2 - 2^{-30}\)
in increments of \(2^{-30}\).

Now, to convert a \mCommand{double} to a fixed-point value, we call
\mCommand[https://www.mathworks.com/help/fixedpoint/ref/embedded.fi.%
html]{fi()}:

\mCommandOutput{09-fixed-point-arithmetic.d/fi-call}

\subsection{Set Fixed-Point Math Settings}

At times, we may wish to change MATLAB's default behavior when
performing fixed-point math.  Say we wanted to have two's complement
overflow as opposed to saturation:

\mCommandOutput{09-fixed-point-arithmetic.d/fi-saturate}

\mCommandOutput{09-fixed-point-arithmetic.d/fi-wrap}

We can also apply these properties to new fixed-point types with
\mCommand[https://www.mathworks.com/help/fixedpoint/ref/embedded.%
fimath.html]{fimath()}:

\mCommandOutput{09-fixed-point-arithmetic.d/fimath-OverflowMode}
\end{document}
