% SPDX-License-Identifier: CC-BY-NC-SA-4.0
%
% 04-func-off.tex -- i'm not doing this again
% Copyright (C) 2024  Jacob Koziej <jacobkoziej@gmail.com>

\documentclass{article}

% SPDX-License-Identifier: CC-BY-NC-SA-4.0
%
% preamble.tex -- document configuration
% Copyright (C) 2024  Jacob Koziej <jacobkoziej@gmail.com>

\usepackage{geometry}
\geometry{
	marginpar=1.8cm,
	paper=b5paper,
}

\usepackage{bookmark}
\usepackage{fancyhdr}
\usepackage{fontawesome}
\usepackage{marginnote}
\usepackage{titling}

\usepackage{mathtools}
\usepackage{unicode-math}
\unimathsetup{
	math-style=ISO,
	warnings-off={
		mathtools-colon,
		mathtools-overbracket,
	},
}

\usepackage{minted}
\setminted{
	breaklines,
	linenos,
	obeytabs,
}

\input{git-hash}

\author{Jacob~Koziej}

\pagestyle{fancy}
\setcounter{secnumdepth}{0}

\fancyhead{}
\fancyhead[L]{%
	\small\slshape%
	ECE 210: MATLAB Seminar -- Signals \& Systems%
}
\fancyhead[R]{%
	\scriptsize\ttfamily%
	\faCodeFork\,\gitHash\gitDirty%
}
\fancyfoot{}
\fancyfoot[L]{%
	\scriptsize\slshape%
	Copyright \copyright\ 2024 \theauthor%
	~---~%
	\ttfamily%
	\href{https://creativecommons.org/licenses/by-nc-sa/4.0/}%
	{CC BY-NC-SA 4.0}%
}
\fancyfoot[R]{%
	\scriptsize\ttfamily%
	\href{https://github.com/jacobkoziej/jk-ece210}%
	{github.com/jacobkoziej/jk-ece210}%
}
\RenewDocumentCommand{\footrule}{}{\rule{\headwidth}{\headrulewidth}}
\RenewDocumentCommand{\footruleskip}{}{1pt}

\NewDocumentCommand{\aside}{m}{%
	\marginnote{%
		\footnotesize%
		\raggedright%
		\textsl{\textbf{Aside---}}%
		\par\noindent%
		#1%
	}
}

\NewDocumentCommand{\footurl}{m}{\footnote{\url{#1}}}

\NewDocumentCommand{\mCommand}{om}{%
	\IfValueTF{#1} {%
		\href{#1}{\mintinline{matlab}{#2}}%
	} {%
		\mintinline{matlab}{#2}%
	}%
}

\NewDocumentCommand{\matrixField}{mmm}{%
	\symrm{M}_{#1 \times #2} (\symbb{#3})%
}

\NewDocumentCommand{\renderTitle}{}{%
	{\noindent\LARGE\scshape\thetitle\vspace{1ex}}%
	\pdfbookmark[1]{\thetitle}{title}%
}

\NewDocumentCommand{\vocab}{m}{\textsl{\textbf{#1}}}

\title{Assignment 04: Func Off}
% SPDX-License-Identifier: CC-BY-NC-SA-4.0
%
% postamble.tex -- document configuration, continued...
% Copyright (C) 2024  Jacob Koziej <jacobkoziej@gmail.com>

\usepackage{hyperref}
\hypersetup{
	hidelinks,
	pdfinfo = {
		Title    = \thetitle,
		Author   = \theauthor,
		Subject  = MATLAB,
		Keywords = {MATLAB, programming},
	},
}


\begin{document}
\renderTitle

\begin{enumerate}[leftmargin=*]
	\item
		Create an \mCommand{ip} anonymous function that performs
		the standard inner product over \(\C\) and an
		\mCommand{ip_norm} anonymous function for its associated
		\(\symrm{L}^2\) norm.

	\item
		Create a \mCommand{gram_schmidt} function.  The input
		should be a matrix of linearly independent columns and
		function handles that define an inner product and norm
		for the Gram–Schmidt process.  The function should
		return a matrix of orthonormal column vectors.

	\item
		Create an \mCommand{isorthogonal} function that accepts
		two vectors and an inner product function handle.  The
		function should return \mCommand{true} if the vectors
		are orthogonal and \mCommand{false} if they are not.
		Due to the numerical instability of the Gram-Schmidt
		process, our orthonormal vectors may not be exactly
		orthogonal.  Take this into account by utilizing the
		\mCommand[https://www.mathworks.com/help/matlab/ref/%
		eps.html]{eps} function.

	\item
		Calculate orthonormal vectors from the following
		orthogonal set:
		\begin{equation}
			\symcal{S}
			=
			\left\{
				\begin{bmatrix}
					1       \\
					\im     \\
					2 - \im \\
					-1      \\
				\end{bmatrix},
				\begin{bmatrix}
					2 + 3\im \\
					3\im     \\
					1 - \im  \\
					2\im     \\
				\end{bmatrix},
				\begin{bmatrix}
					-1 + 7\im \\
					6 + 10\im \\
					11 - 4\im \\
					3 + 4\im  \\
				\end{bmatrix}
			\right\}
		\end{equation}
		and store them into matrix \mCommand{U}.
\end{enumerate}
\end{document}
