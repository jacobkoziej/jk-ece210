% SPDX-License-Identifier: CC-BY-NC-SA-4.0
%
% 06-four-year-transform.tex -- in-n-out the transform domain
% Copyright (C) 2024  Jacob Koziej <jacobkoziej@gmail.com>

\documentclass{article}

% SPDX-License-Identifier: CC-BY-NC-SA-4.0
%
% preamble.tex -- document configuration
% Copyright (C) 2024  Jacob Koziej <jacobkoziej@gmail.com>

\usepackage{geometry}
\geometry{
	paper=b5paper,
}

\usepackage{bookmark}
\usepackage{fancyhdr}
\usepackage{fontawesome}
\usepackage{titling}

\usepackage{mathtools}
\usepackage{unicode-math}
\unimathsetup{
	math-style=ISO,
	warnings-off={
		mathtools-colon,
		mathtools-overbracket,
	},
}

\usepackage{minted}
\setminted{
	breaklines,
	linenos,
	obeytabs,
}

\input{git-hash}

\author{Jacob~Koziej}

\pagestyle{fancy}
\setcounter{secnumdepth}{0}

\fancyhead{}
\fancyhead[L]{%
	\small\slshape%
	ECE 210: MATLAB Seminar -- Signals \& Systems%
}
\fancyhead[R]{%
	\scriptsize\ttfamily%
	\faCodeFork\,\gitHash\gitDirty%
}
\fancyfoot{}
\fancyfoot[L]{%
	\scriptsize\slshape%
	Copyright \copyright\ 2024 \theauthor%
	~---~%
	\ttfamily%
	\href{https://creativecommons.org/licenses/by-nc-sa/4.0/}%
	{CC BY-NC-SA 4.0}%
}
\fancyfoot[R]{%
	\scriptsize\ttfamily%
	\href{https://github.com/jacobkoziej/jk-ece210}%
	{github.com/jacobkoziej/jk-ece210}%
}
\RenewDocumentCommand{\footrule}{}{\rule{\headwidth}{\headrulewidth}}
\RenewDocumentCommand{\footruleskip}{}{1pt}

\NewDocumentCommand{\footurl}{m}{\footnote{\url{#1}}}

\NewDocumentCommand{\mCommand}{om}{%
	\IfValueTF{#1} {%
		\href{#1}{\mintinline{matlab}{#2}}%
	} {%
		\mintinline{matlab}{#2}%
	}%
}

\NewDocumentCommand{\renderTitle}{}{%
	{\noindent\LARGE\scshape\thetitle\vspace{1ex}}%
	\pdfbookmark[1]{\thetitle}{title}%
}

\NewDocumentCommand{\vocab}{m}{\textsl{\textbf{#1}}}

\title{Assignment 06: Four-Year Transform}
% SPDX-License-Identifier: CC-BY-NC-SA-4.0
%
% postamble.tex -- document configuration, continued...
% Copyright (C) 2024  Jacob Koziej <jacobkoziej@gmail.com>

\usepackage{hyperref}
\hypersetup{
	hidelinks,
	pdfinfo = {
		Title    = \thetitle,
		Author   = \theauthor,
		Subject  = MATLAB,
		Keywords = {MATLAB, programming},
	},
}


\begin{document}
\renderTitle

Assume for this assignment that we're sampling at 96 kHz.

\begin{enumerate}[leftmargin=*]
	\item
		Define your own anonymous \mCommand{db2mag()} function
		so that you can easily generate signals at different dB
		levels.

	\item
		Generate a signal with 192k samples with frequencies at
		-20.48~kHz, -360~Hz, 996~Hz, and 19.84~kHz at 14~dB,
		-10~dB, 0~dB, and 2~dB respectively.  Next, add -10~dB
		of white noise with \mCommand[https://www.mathworks.%
		com/help/matlab/ref/randn.html]{randn()}.  Finally, take
		the DFT and plot its magnitude on the dB scale.

	\item
		Given the following digital filter:
		\begin{equation}
			H(z)
			=
			0.53
			\frac
			{
				(z - (0.76 \pm \im 0.64))
				(z - (0.69 \pm \im 0.71))
				(z - (0.82 \pm \im 0.57))
			}
			{
				(z - (0.57 \pm \im 0.78))
				(z - (0.85 \pm \im 0.48))
				(z - 0.24)
				(z - 0.64)
			}
		\end{equation}
		Generate a pole-zero plot along with plots for the
		magnitude and phase response.  For the latter two, look
		at \mCommand[https://www.mathworks.com/help/signal/ref/%
		freqz.html]{freqz()}'s documentation to figure out how
		to get the response of this filter over physical
		frequencies.
\end{enumerate}
\end{document}
