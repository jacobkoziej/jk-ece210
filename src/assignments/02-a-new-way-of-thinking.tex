% SPDX-License-Identifier: CC-BY-NC-SA-4.0
%
% 02-a-new-way-of-thinking.tex -- vectorization
% Copyright (C) 2024  Jacob Koziej <jacobkoziej@gmail.com>

\documentclass{article}

% SPDX-License-Identifier: CC-BY-NC-SA-4.0
%
% preamble.tex -- document configuration
% Copyright (C) 2024  Jacob Koziej <jacobkoziej@gmail.com>

\usepackage{geometry}
\geometry{
	marginpar=1.8cm,
	paper=b5paper,
}

\usepackage{bookmark}
\usepackage{fancyhdr}
\usepackage{fontawesome}
\usepackage{marginnote}
\usepackage{titling}

\usepackage{mathtools}
\usepackage{unicode-math}
\unimathsetup{
	math-style=ISO,
	warnings-off={
		mathtools-colon,
		mathtools-overbracket,
	},
}

\usepackage{minted}
\setminted{
	breaklines,
	linenos,
	obeytabs,
}

\input{git-hash}

\author{Jacob~Koziej}

\pagestyle{fancy}
\setcounter{secnumdepth}{0}

\fancyhead{}
\fancyhead[L]{%
	\small\slshape%
	ECE 210: MATLAB Seminar -- Signals \& Systems%
}
\fancyhead[R]{%
	\scriptsize\ttfamily%
	\faCodeFork\,\gitHash\gitDirty%
}
\fancyfoot{}
\fancyfoot[L]{%
	\scriptsize\slshape%
	Copyright \copyright\ 2024 \theauthor%
	~---~%
	\ttfamily%
	\href{https://creativecommons.org/licenses/by-nc-sa/4.0/}%
	{CC BY-NC-SA 4.0}%
}
\fancyfoot[R]{%
	\scriptsize\ttfamily%
	\href{https://github.com/jacobkoziej/jk-ece210}%
	{github.com/jacobkoziej/jk-ece210}%
}
\RenewDocumentCommand{\footrule}{}{\rule{\headwidth}{\headrulewidth}}
\RenewDocumentCommand{\footruleskip}{}{1pt}

\NewDocumentCommand{\aside}{m}{%
	\marginnote{%
		\footnotesize%
		\raggedright%
		\textsl{\textbf{Aside---}}%
		\par\noindent%
		#1%
	}
}

\NewDocumentCommand{\footurl}{m}{\footnote{\url{#1}}}

\NewDocumentCommand{\mCommand}{om}{%
	\IfValueTF{#1} {%
		\href{#1}{\mintinline{matlab}{#2}}%
	} {%
		\mintinline{matlab}{#2}%
	}%
}

\NewDocumentCommand{\matrixField}{mmm}{%
	\symrm{M}_{#1 \times #2} (\symbb{#3})%
}

\NewDocumentCommand{\renderTitle}{}{%
	{\noindent\LARGE\scshape\thetitle\vspace{1ex}}%
	\pdfbookmark[1]{\thetitle}{title}%
}

\NewDocumentCommand{\vocab}{m}{\textsl{\textbf{#1}}}

\title{Assignment 02: A New Way of Thinking}
% SPDX-License-Identifier: CC-BY-NC-SA-4.0
%
% postamble.tex -- document configuration, continued...
% Copyright (C) 2024  Jacob Koziej <jacobkoziej@gmail.com>

\usepackage{hyperref}
\hypersetup{
	hidelinks,
	pdfinfo = {
		Title    = \thetitle,
		Author   = \theauthor,
		Subject  = MATLAB,
		Keywords = {MATLAB, programming},
	},
}


\begin{document}
\renderTitle

\begin{enumerate}[leftmargin=*]
	\item
		Create the following vectors:
		\begin{equation}
			\mat{u}
			=
			[-4, -2, 0, 2, 4]
		\end{equation}
		\begin{equation}
			\mat{v}
			=
			\left[
				0,
				\frac{\cpi}{4},
				\frac{\cpi}{2},
				\frac{3\cpi}{4},
				\cpi
			\right]
		\end{equation}

	\item
		Calculate \(10!\) using \mCommand[https://www.%
		mathworks.com/help/matlab/ref/prod.html]{prod()} and
		store it into variable \mCommand{f}.

	\item
		Create the following matrices:
		\begin{enumerate}
			\item
				\begin{equation}
					\mat{A}
					=
					\begin{bmatrix}
						1 & 0 & 0 & 0 \\
						0 & 0 & 1 & 0 \\
					\end{bmatrix}
				\end{equation}
				\hint{Look at the documentation for the
				\mCommand[https://www.mathworks.com/%
				help/matlab/ref/zeros.html]{zeros()}
				function.}

			\item
				\begin{equation}
					\mat{B}
					=
					\begin{bmatrix}
						1 &  9 & 2 & 10 \\
						3 & 11 & 4 & 12 \\
						5 & 13 & 6 & 14 \\
						7 & 15 & 8 & 16 \\
					\end{bmatrix}
				\end{equation}
				\hint{MATLAB stores matrices in column
				major order.  An intermediate
				\mCommand{reshape()} with a transpose
				can give you a \enquote{nice} matrix to
				play with.}
		\end{enumerate}
\end{enumerate}
\end{document}
