% SPDX-License-Identifier: CC-BY-NC-SA-4.0
%
% 03-amogus.tex -- when the assignment sus
% Copyright (C) 2024  Jacob Koziej <jacobkoziej@gmail.com>

\documentclass{article}

% SPDX-License-Identifier: CC-BY-NC-SA-4.0
%
% preamble.tex -- document configuration
% Copyright (C) 2024  Jacob Koziej <jacobkoziej@gmail.com>

\usepackage{geometry}
\geometry{
	marginpar=1.8cm,
	paper=b5paper,
}

\usepackage{bookmark}
\usepackage{fancyhdr}
\usepackage{fontawesome}
\usepackage{marginnote}
\usepackage{titling}

\usepackage{mathtools}
\usepackage{unicode-math}
\unimathsetup{
	math-style=ISO,
	warnings-off={
		mathtools-colon,
		mathtools-overbracket,
	},
}

\usepackage{minted}
\setminted{
	breaklines,
	linenos,
	obeytabs,
}

\input{git-hash}

\author{Jacob~Koziej}

\pagestyle{fancy}
\setcounter{secnumdepth}{0}

\fancyhead{}
\fancyhead[L]{%
	\small\slshape%
	ECE 210: MATLAB Seminar -- Signals \& Systems%
}
\fancyhead[R]{%
	\scriptsize\ttfamily%
	\faCodeFork\,\gitHash\gitDirty%
}
\fancyfoot{}
\fancyfoot[L]{%
	\scriptsize\slshape%
	Copyright \copyright\ 2024 \theauthor%
	~---~%
	\ttfamily%
	\href{https://creativecommons.org/licenses/by-nc-sa/4.0/}%
	{CC BY-NC-SA 4.0}%
}
\fancyfoot[R]{%
	\scriptsize\ttfamily%
	\href{https://github.com/jacobkoziej/jk-ece210}%
	{github.com/jacobkoziej/jk-ece210}%
}
\RenewDocumentCommand{\footrule}{}{\rule{\headwidth}{\headrulewidth}}
\RenewDocumentCommand{\footruleskip}{}{1pt}

\NewDocumentCommand{\aside}{m}{%
	\marginnote{%
		\footnotesize%
		\raggedright%
		\textsl{\textbf{Aside---}}%
		\par\noindent%
		#1%
	}
}

\NewDocumentCommand{\footurl}{m}{\footnote{\url{#1}}}

\NewDocumentCommand{\mCommand}{om}{%
	\IfValueTF{#1} {%
		\href{#1}{\mintinline{matlab}{#2}}%
	} {%
		\mintinline{matlab}{#2}%
	}%
}

\NewDocumentCommand{\matrixField}{mmm}{%
	\symrm{M}_{#1 \times #2} (\symbb{#3})%
}

\NewDocumentCommand{\renderTitle}{}{%
	{\noindent\LARGE\scshape\thetitle\vspace{1ex}}%
	\pdfbookmark[1]{\thetitle}{title}%
}

\NewDocumentCommand{\vocab}{m}{\textsl{\textbf{#1}}}

\title{Assignment 03: Amogus}
% SPDX-License-Identifier: CC-BY-NC-SA-4.0
%
% postamble.tex -- document configuration, continued...
% Copyright (C) 2024  Jacob Koziej <jacobkoziej@gmail.com>

\usepackage{hyperref}
\hypersetup{
	hidelinks,
	pdfinfo = {
		Title    = \thetitle,
		Author   = \theauthor,
		Subject  = MATLAB,
		Keywords = {MATLAB, programming},
	},
}


\begin{document}
\renderTitle

\begin{wrapfigure}{l}{0.2\textwidth}
	\resizebox{0.2\textwidth}{!}{
		\begin{tikzpicture}
			\amongUsIII{red}{blue}
		\end{tikzpicture}
	}
\end{wrapfigure}

Oh no!  There's an imposter \textit{\textsc{Among Us}} \emoji{flushed-face}

Fortunately for us, we have the power of MATLAB on our side and can
\emph{calculate} millions of scenarios to see how likely we are to lose
to the imposter.  To make sure that we can find out our odds of survival
before the imposter kills us, we'll need to vectorize and use our newly
learned methods of indexing!

Since it's difficult to model crewmate and imposter behavior, we'll have
to make a few simplifications to ensure that we can complete our
simulation in time.  The rules for our simulation are as follows:

\begin{itemize}[leftmargin=*]
	\item
		The imposter's \emph{sus} ability is the sum of two
		rolls of a two sided die.

	\item
		Each crewmate's \emph{sus resistance} is the roll of a
		four sided die.

	\item
		An imposter can kill a crewmate only if their sus ability is
		greater than the crewmate's sus resistance.

	\item
		There is one imposter, six crewmates, and twelve rounds.

	\item
		Each round, the imposter will target one crewmate at
		random.  If the crewmate has already been killed or has
		a greater sus resistance score, nothing happens.

	\item
		A loss occurs if one or fewer crewmates remain after the
		twelve rounds.
\end{itemize}

To ensure that we have accurate results, we'll run our simulation for a
million iterations.  Since we're dealing with so many iterations, please
include the following starter code to have deterministic results:

\begin{minted}{matlab}
ITERATIONS = 1e6;

CREWMATES = 6;
ROUNDS = 12;

CREWMATE_SIDES = 4;
IMPOSTER_ROLLS = 2;
IMPOSTER_SIDES = 2;

rng(0x73757300);
\end{minted}

Here, \mCommand[https://www.mathworks.com/help/matlab/ref/rng.html]{%
rng()} sets MATLAB's random number generator seed so that we can
reproduce our calls to random functions.

\begin{enumerate}[leftmargin=*]
	\item
		For this part, utilize \mCommand{randi()} to generate
		matrices for our \enquote{random} events. Create a
		\mCommand{crewmates} matrix with each crewmate's sus
		resistance, a \mCommand{sus} matrix with the imposter's
		sus ability, and  a \mCommand{targets} matrix with the
		imposter's target for each round.

		\hint{For each random event, add a dimension.}

		\hint{For the sus ability, add a dimension for rolls and
		\mCommand[https://www.mathworks.com/help/matlab/ref/%
		sum.html]{sum()} across it.}

	\item
		Now generate a logical \mCommand{kills} matrix that will
		represent all crewmates, the imposter can that can kill
		in twelve rounds.

		\hint{The \mCommand{targets} matrix contains the
		crewmate the imposter will target.  You can treat the
		crewmate and the round as a subscript index.  If you
		convert these to linear indices, you can easily create
		this logical matrix given an initial \mCommand[https://%
		www.mathworks.com/help/matlab/ref/false.html]{false()}
		matrix.}
\end{enumerate}
\end{document}
