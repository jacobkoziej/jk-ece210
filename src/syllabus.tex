% SPDX-License-Identifier: CC-BY-NC-SA-4.0
%
% syllabus.tex -- class expectations
% Copyright (C) 2024  Jacob Koziej <jacobkoziej@gmail.com>

\documentclass{article}

% SPDX-License-Identifier: CC-BY-NC-SA-4.0
%
% preamble.tex -- document configuration
% Copyright (C) 2024  Jacob Koziej <jacobkoziej@gmail.com>

\usepackage{geometry}
\geometry{
	paper=b5paper,
}

\usepackage{bookmark}
\usepackage{fancyhdr}
\usepackage{fontawesome}
\usepackage{titling}

\usepackage{mathtools}
\usepackage{unicode-math}
\unimathsetup{
	math-style=ISO,
	warnings-off={
		mathtools-colon,
		mathtools-overbracket,
	},
}

\usepackage{minted}
\setminted{
	breaklines,
	linenos,
	obeytabs,
}

\input{git-hash}

\author{Jacob~Koziej}

\pagestyle{fancy}
\setcounter{secnumdepth}{0}

\fancyhead{}
\fancyhead[L]{%
	\small\slshape%
	ECE 210: MATLAB Seminar -- Signals \& Systems%
}
\fancyhead[R]{%
	\scriptsize\ttfamily%
	\faCodeFork\,\gitHash\gitDirty%
}
\fancyfoot{}
\fancyfoot[L]{%
	\scriptsize\slshape%
	Copyright \copyright\ 2024 \theauthor%
	~---~%
	\ttfamily%
	\href{https://creativecommons.org/licenses/by-nc-sa/4.0/}%
	{CC BY-NC-SA 4.0}%
}
\fancyfoot[R]{%
	\scriptsize\ttfamily%
	\href{https://github.com/jacobkoziej/jk-ece210}%
	{github.com/jacobkoziej/jk-ece210}%
}
\RenewDocumentCommand{\footrule}{}{\rule{\headwidth}{\headrulewidth}}
\RenewDocumentCommand{\footruleskip}{}{1pt}

\NewDocumentCommand{\footurl}{m}{\footnote{\url{#1}}}

\NewDocumentCommand{\mCommand}{om}{%
	\IfValueTF{#1} {%
		\href{#1}{\mintinline{matlab}{#2}}%
	} {%
		\mintinline{matlab}{#2}%
	}%
}

\NewDocumentCommand{\renderTitle}{}{%
	{\noindent\LARGE\scshape\thetitle\vspace{1ex}}%
	\pdfbookmark[1]{\thetitle}{title}%
}

\NewDocumentCommand{\vocab}{m}{\textsl{\textbf{#1}}}

\title{Syllabus}
% SPDX-License-Identifier: CC-BY-NC-SA-4.0
%
% postamble.tex -- document configuration, continued...
% Copyright (C) 2024  Jacob Koziej <jacobkoziej@gmail.com>

\usepackage{hyperref}
\hypersetup{
	hidelinks,
	pdfinfo = {
		Title    = \thetitle,
		Author   = \theauthor,
		Subject  = MATLAB,
		Keywords = {MATLAB, programming},
	},
}


\NewDocumentCommand{\entry}{m}{\textsl{\textbf{#1}}}

\begin{document}
\renderTitle

\scshape
\slshape

\begin{Centering}
\noindent
The Cooper Union for the Advancement of Science and Art

\noindent
Albert Nerken School of Engineering

\noindent
Department of Electrical Engineering

\noindent
Spring 2024

\leavevmode
\newline

\noindent
ECE 210-A: MATLAB Seminar -- Signals \& Systems

\end{Centering}

\normalfont

\section{Course Overview}

\entry{Catalog Description:}

\begin{displayquote}
	A weekly hands-on, interactive seminar that introduces students
	to MATLAB, in general, and the Signal Processing Toolbox in
	particular.  Students explore scientific computation and
	scientific visualization with MATLAB.  Concepts of signal
	processing and system analysis that are presented in ECE 211 or
	other introductory courses on the subject are reinforced through
	a variety of demonstrations and exercises.  It is strongly
	encouraged for students taking a first course in signals and
	systems, or for students expecting to use MATLAB in projects or
	courses.

	\footnotesize

	\noindent
	\entry{Pre-requisite:} \textsc{\textsl{MA 113 -- Calculus II}}

	\noindent
	\entry{Co-requisite:} \textsc{\textsl{ECE 211 -- Signals
	Processing}}

	\noindent
	\textsl{\textbf{tldr:} we'll be learning how to use MATLAB from
	a signals processing perspective}
\end{displayquote}

\noindent
\entry{Instructor:} Jacob Koziej (EE '25)

\noindent
\entry{Location:} 41 Cooper Square, Room 604

\noindent
\entry{Time:} Wednesdays 12:00--12:50

\noindent
\entry{Credits:} Zero\footnote{\blockquote[Fred L. Fontaine]{Secret one
credit class because I couldn't make Signals four credits.}}

\noindent
\entry{Office Hours:} On request

\noindent
\entry{Preferred Contact Method:} Email

\noindent
\entry{Course Website:} \url{https://github.com/jacobkoziej/jk-ece210}
\end{document}
