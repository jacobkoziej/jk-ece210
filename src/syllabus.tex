% SPDX-License-Identifier: CC-BY-NC-SA-4.0
%
% syllabus.tex -- class expectations
% Copyright (C) 2024  Jacob Koziej <jacobkoziej@gmail.com>

\documentclass{article}

% SPDX-License-Identifier: CC-BY-NC-SA-4.0
%
% preamble.tex -- document configuration
% Copyright (C) 2024  Jacob Koziej <jacobkoziej@gmail.com>

\usepackage{geometry}
\geometry{
	marginpar=1.8cm,
	paper=b5paper,
}

\usepackage{bookmark}
\usepackage{fancyhdr}
\usepackage{fontawesome}
\usepackage{marginnote}
\usepackage{titling}

\usepackage{mathtools}
\usepackage{unicode-math}
\unimathsetup{
	math-style=ISO,
	warnings-off={
		mathtools-colon,
		mathtools-overbracket,
	},
}

\usepackage{minted}
\setminted{
	breaklines,
	linenos,
	obeytabs,
}

\input{git-hash}

\author{Jacob~Koziej}

\pagestyle{fancy}
\setcounter{secnumdepth}{0}

\fancyhead{}
\fancyhead[L]{%
	\small\slshape%
	ECE 210: MATLAB Seminar -- Signals \& Systems%
}
\fancyhead[R]{%
	\scriptsize\ttfamily%
	\faCodeFork\,\gitHash\gitDirty%
}
\fancyfoot{}
\fancyfoot[L]{%
	\scriptsize\slshape%
	Copyright \copyright\ 2024 \theauthor%
	~---~%
	\ttfamily%
	\href{https://creativecommons.org/licenses/by-nc-sa/4.0/}%
	{CC BY-NC-SA 4.0}%
}
\fancyfoot[R]{%
	\scriptsize\ttfamily%
	\href{https://github.com/jacobkoziej/jk-ece210}%
	{github.com/jacobkoziej/jk-ece210}%
}
\RenewDocumentCommand{\footrule}{}{\rule{\headwidth}{\headrulewidth}}
\RenewDocumentCommand{\footruleskip}{}{1pt}

\NewDocumentCommand{\aside}{m}{%
	\marginnote{%
		\footnotesize%
		\raggedright%
		\textsl{\textbf{Aside---}}%
		\par\noindent%
		#1%
	}
}

\NewDocumentCommand{\footurl}{m}{\footnote{\url{#1}}}

\NewDocumentCommand{\mCommand}{om}{%
	\IfValueTF{#1} {%
		\href{#1}{\mintinline{matlab}{#2}}%
	} {%
		\mintinline{matlab}{#2}%
	}%
}

\NewDocumentCommand{\matrixField}{mmm}{%
	\symrm{M}_{#1 \times #2} (\symbb{#3})%
}

\NewDocumentCommand{\renderTitle}{}{%
	{\noindent\LARGE\scshape\thetitle\vspace{1ex}}%
	\pdfbookmark[1]{\thetitle}{title}%
}

\NewDocumentCommand{\vocab}{m}{\textsl{\textbf{#1}}}

\title{Syllabus}
% SPDX-License-Identifier: CC-BY-NC-SA-4.0
%
% postamble.tex -- document configuration, continued...
% Copyright (C) 2024  Jacob Koziej <jacobkoziej@gmail.com>

\usepackage{hyperref}
\hypersetup{
	hidelinks,
	pdfinfo = {
		Title    = \thetitle,
		Author   = \theauthor,
		Subject  = MATLAB,
		Keywords = {MATLAB, programming},
	},
}


\NewDocumentCommand{\entry}{m}{\textsl{\textbf{#1}}}

\begin{document}
\renderTitle

\scshape
\slshape

\begin{Centering}
\noindent
The Cooper Union for the Advancement of Science and Art

\noindent
Albert Nerken School of Engineering

\noindent
Department of Electrical Engineering

\noindent
Spring 2024

\leavevmode
\newline

\noindent
ECE 210-A: MATLAB Seminar -- Signals \& Systems

\end{Centering}

\normalfont

\section{Course Overview}

\entry{Catalog Description:}

\begin{displayquote}
	A weekly hands-on, interactive seminar that introduces students
	to MATLAB, in general, and the Signal Processing Toolbox in
	particular.  Students explore scientific computation and
	scientific visualization with MATLAB.  Concepts of signal
	processing and system analysis that are presented in ECE 211 or
	other introductory courses on the subject are reinforced through
	a variety of demonstrations and exercises.  It is strongly
	encouraged for students taking a first course in signals and
	systems, or for students expecting to use MATLAB in projects or
	courses.

	\footnotesize

	\noindent
	\entry{Pre-requisite:} \textsc{\textsl{MA 113 -- Calculus II}}

	\noindent
	\entry{Co-requisite:} \textsc{\textsl{ECE 211 -- Signals
	Processing}}

	\noindent
	\textsl{\textbf{tldr:} we'll be learning how to use MATLAB from
	a signals processing perspective}
\end{displayquote}

\noindent
\entry{Instructor:} Jacob Koziej (EE '25)

\noindent
\entry{Location:} 41 Cooper Square, Room 802

\noindent
\entry{Time:} Mondays 15:00--15:50

\noindent
\entry{Credits:} Zero\footnote{\blockquote[Fred L. Fontaine]{Secret one
credit class because I couldn't make Signals four credits.}}

\noindent
\entry{Office Hours:} On request

\noindent
\entry{Preferred Contact Method:} Email

\noindent
\entry{Course Website:} \url{https://github.com/jacobkoziej/jk-ece210}

\section{Prerequisite Skills}

Although not explicit prerequisites, we will \emph{heavily} utilize
concepts from linear algebra and an introductory programming course.
After all, MATrix LABratory has \emph{matrix} in the name and is
synonymous with the word \emph{code}.  We will also deal with
differentiation, integration, summation, complex numbers, and numerical
representation.  If you are uncomfortable with \textbf{any} of these
topics, I highly suggest visiting an ARC tutor to get caught up to
speed.

\section{Course Goals}

The ultimate goal of this course is to familiarize you enough with
MATLAB so that you can perform calculations expected of every EE to know
by heart.  By the end of the semester, you should:

\begin{itemize}
	\item
		Write \emph{clean \& effective} MATLAB code.

	\item
		Vectorize numerical operations.

	\item
		Visualize data in a meaningful manner.

	\item
		Mitigate round-off error.

	\item
		Work in the z, s, and frequency domains.

	\item
		Design filters.
\end{itemize}

\noindent
This course will also achieve the following ABET outcomes:

\begin{enumerate}
	\slshape

	\item
		An ability to identify, formulate, and solve complex
		engineering problems by applying principles of
		engineering, science, and mathematics.

	\setcounter{enumi}{5}

	\item
		An ability to develop and conduct appropriate
		experimentation, analyze and interpret data, and use
		engineering judgment to draw conclusions.

	\item
		An ability to acquire and apply new knowledge as needed,
		using appropriate learning strategies.
\end{enumerate}

\section{Policy}

Overall, I am a \emph{very} understanding person, but please don't force
my hand.

\subsection{Ask Questions}

\begin{displayquote}[Confucius]
	The man who asks a question is a fool for a minute, the man who
	does not ask is a fool for life.
\end{displayquote}

\noindent
At times, my mind moves too fast for my own good, and I may \emph{skip}
steps.  If at any moment you are confused for any reason, \textbf{stop
me}.  I can guarantee you are not the only person in the room.  If
you're so confused you can't compose a meaningful question,
\textbf{please reach out}, and we'll identify the problem together.

\subsection{Academic Integrity}

I will not tolerate cheating of \textbf{any} kind.  This is such a
low-stakes course that if you cheat, I will seriously be disappointed
and question how you're operating in your more rigorous courses.
Generative AI is also strictly prohibited.  The whole point of this
course is to \emph{learn} how to write good MATLAB code, not to mention
that the code quality of such tools is still quite shitty.  You'll have
plenty of time in your career to toy with these tools.  \emph{If} you
get stumped and receive help from someone or find code online,
\textbf{cite the source(s)} in your submission.  Ultimately, if I
suspect you have cheated, I must, by school policy, report you to the
Dean's office for a formal investigation.

\subsection{Attendance}

Attendance is mandatory.  You do not need to inform me if you can't make
a lecture, but if I notice you have not attended my lectures regularly,
I will reach out to find out why.  Although I am creating supplementary
lecture material, it will not cover everything I cover during a lecture.

\subsection{Course Material}

All course material is freely available on the course website.  The page
will always have the most up-to-date course material.  The bottom right
of each PDF has a clickable link to the course website, and the top
right also has a clickable Git commit hash associated with the generated
PDF.  It is essential for assignments we all operate on the \textbf{%
same} commit hash.  \emph{If,} for any reason, a change is necessary, I
will inform you of the new commit hash.  \emph{However,} I will not
penalize you for using previous commit hashes due to an error on my end.

\subsection{Grading}

All assignments are pass/fail.  If you fail, you will receive detailed
feedback as to \emph{why.}  \textbf{I do not allow re-submissions.}
There are six \enquote{core} assignments you need to successfully
complete to pass this course.  If you fail one or more, you will need to
complete an equal amount of additional assignments I create that delve
into topics that interest \emph{me} to \enquote{make up} these failures.
You can expect about six of these.

You will have \emph{at least} one week to complete each assignment, and
submissions are to be placed in my mailbox at 41 Cooper Square by
midnight.  \textbf{I do not accept late work.}\footnote{Late in my book
means when I get to grading.}  If you need an extension, \textbf{reach
out to me.}  I also expect all submissions to follow my submission
guidelines,\footurl{https://github.com/jacobkoziej/jk-ece210/blob/%
build/submissions.pdf} as I \underline{will not grade} work that does
not meet these expectations!

\section{Student Resources}

\entry{Accommodations:} \url{https://cooper.edu/students/student-%
affairs/disability}

\

\noindent
\entry{Mental Health Services:} \url{https://cooper.edu/students/%
student-affairs/health/counseling}

\

\noindent
\entry{Title IX Policy:} \url{https://cooper.edu/sites/default/files/%
uploads/assets/site/files/2020/Cooper-Union-Policy-Upholding-Human-%
Rights-Title-IX-Protections.pdf}
\end{document}
